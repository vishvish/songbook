%% `master.tex'---Balham Ukulele Society Songbook. 
%% The master songbook for Balham Ukulele Society.
%% 
%% Copyright (C) 2012 Balham Ukulele Society
%% Maintainer Vish Vishvanath <vish.vishvanath@gmail.com>--author-maintained. 
%% 
\def\fileversion{0.1} \def\filedate{23/03/2012} 
%% 
%% This file, the Balham Ukulele Society Songbook, is licensed under 
%% a Creative Commons Attribution-NonCommercial-ShareAlike 3.0 Unported License
%% http://creativecommons.org/licenses/by-nc-sa/3.0/

% Balham Ukulele Society Songbook
% Master A4 version


% The two lines below configure whether the document is one-sided or two-sided. Uncomment whichever you prefer.
% \documentclass[11pt,a4paper, oneside]{book}
\documentclass[11pt,a4paper]{book}

% Beware of modifying this - you could easily break the layout.
\usepackage[top=1.5cm, bottom=1in, left=1in, right=2in, textwidth=15cm, textheight=26cm, marginparwidth=0in]{geometry}

\usepackage[parfill]{parskip}
\usepackage{amssymb}

% Mini tables of content
\usepackage{minitoc}

% add some colour
\usepackage[usenames,dvipsnames]{xcolor}

% links
\usepackage[colorlinks=true, pdfstartview=FitV, linkcolor=blue, citecolor=blue, urlcolor=blue]{hyperref}

% need linenumbers so we can keep track of the song while playing in the dimly-lit pub
\usepackage[pagewise]{lineno}
\renewcommand\linenumberfont{\color{SkyBlue}\small}


% Set up the fonts
\usepackage{fontspec,xltxtra,xunicode}
\defaultfontfeatures{Mapping=tex-text} % converts LaTeX specials (``quotes'' --- dashes etc.) to unicode

% Some fonts to use if working on a Mac
\setromanfont [Scale=1.1, Ligatures={Common}, Numbers={OldStyle}]{Minion Pro}
% \setromanfont [Scale=0.9, Ligatures={Common}, Numbers={OldStyle}]{Myriad Pro}
% \setromanfont [Scale=0.9, Ligatures={Common}, Numbers={OldStyle}]{Book Antiqua}
% \setromanfont [Scale=0.9, Ligatures={Common}, Numbers={OldStyle}]{Candara}
% \setromanfont [Scale=0.9, Ligatures={Common}, Numbers={OldStyle}]{Cambria}
% \setromanfont [Scale=0.9, Ligatures={Common}, Numbers={OldStyle}]{Chaparral Pro}
% \setromanfont [Scale=0.9, Ligatures={Common}, Numbers={OldStyle}]{Corbel}

% \setmonofont[Scale=0.8]{Monaco} 
% \setsansfont[Scale=0.9]{Optima}

% Custom ampersand
\newcommand{\amper}{{\fontspec[Scale=.95]{Minion Pro}\selectfont\itshape\&}}

% distance between margin and body
\setlength{\marginparsep}{1in}

% linespacing
\linespread{1.9}

% chords
\usepackage{gchords}

% Ukulele Settings
\renewcommand\strings{4}          % number of strings on your guitar
\renewcommand\numfrets{5}         % length (no of frets) of a diagram

% defines the chord above the lyrics
\renewcommand{\upchord}[1]{%
    \settowidth{\cwidth}{#1}%
    \raisebox{15pt}{\color{Gray}#1}\hspace{-\cwidth}%
}

\newcommand\mychords{
	\def\chordsize{2.5mm}   % distance between two frets (and two strings)
	\font\fingerfont=cmr5  % font used for numbering fingers
	% \font\fingerfont=cmmi5   % font used for numbering fingers
	\font\namefont="Myriad Pro"    % font used for labeling of the chord
	\font\fretposfont=cmr7  % font used for the fret position marker
	% \def\dampsymbol{$\scriptstyle\times$} %  `damp this string' marker
	\def\dampsymbol{{\tiny$\scriptstyle\times$}} %  `damp this string' marker
}
\mychords

\renewcommand\yoff{3}
\renewcommand\fingsiz{1.6}

% Begin Chord definitions

% Major chords
\newcommand{\Amajor}{\marginpar{\chord{t}{4,3,2,2}{A}}}
\newcommand{\Bmajor}{\marginpar{\chord{t}{4,3,2,2}{B}}}
\newcommand{\Cmajor}{\marginpar{\chord{t}{o,o,o,3}{C}}}
\newcommand{\Dmajor}{\marginpar{\chord{t}{2,2,2,n}{D}}}
\newcommand{\Emajor}{\marginpar{\chord{t}{4,4,4,2}{E}}}
\newcommand{\Fmajor}{\marginpar{\chord{t}{4,3,2,2}{F}}}
\newcommand{\Gmajor}{\marginpar{\chord{t}{o,2,3,2}{G}}}

% Minor chords
\newcommand{\Aminor}{\marginpar{\chord{t}{2,o,o,o}{A\large{m}}}}
\newcommand{\Bminor}{\marginpar{\chord{t}{3,5,2,1}{B\large{m}}}}
\newcommand{\Cminor}{\marginpar{\chord{t}{3,5,2,1}{C\large{m}}}}
\newcommand{\Dminor}{\marginpar{\chord{t}{2,2,1,o}{D\large{m}}}}
\newcommand{\Eminor}{\marginpar{\chord{t}{3,5,2,1}{E\large{m}}}}
\newcommand{\Fminor}{\marginpar{\chord{t}{3,5,2,1}{F\large{m}}}}
\newcommand{\Gminor}{\marginpar{\chord{t}{3,5,2,1}{G\large{m}}}}

% Seventh chords
\newcommand{\CmajorSeven}{\marginpar{\chord{t}{o,o,o,2}{Cmaj\large{7}}}}

\newcommand{\EminorSeven}{\marginpar{\chord{t}{o,2,o,2}{E\large{m}7}}}

\newcommand{\Aseven}{\marginpar{\chord{t}{o,1,o,o}{A\large{7}}}}
\newcommand{\Cseven}{\marginpar{\chord{t}{o,o,o,1}{C\large{7}}}}
\newcommand{\Dseven}{\marginpar{\chord{t}{2,2,3,3}{D\large{7}}}}
\newcommand{\Eseven}{\marginpar{\chord{t}{1,2,o,2}{E\large{7}}}}
\newcommand{\Gseven}{\marginpar{\chord{t}{o,2,1,2}{G\large{7}}}}

% Other chords
\newcommand{\Bflat}{\marginpar{\chord{t}{3,2,1,1}{B$\flat$}}}

\newcommand{\BflatDimishedSeven}{\marginpar{\chord{t}{3,2,1,1}{B$\flat$dim\large{7}}}}

\newcommand{\Caugmented}{\marginpar{\chord{t}{1,o,o,3}{C+}}}
\newcommand{\Gaugmented}{\marginpar{\chord{t}{4,3,3,2}{G+}}}

% End Chord definitions

% ------------------- Title and Author -----------------------------
\title{Balham Ukulele Society Songbook}
\author{Maintainer: Vish Vishvanath}
\begin{document}

\pagenumbering{roman}
\pagestyle{plain}
\newgeometry{top=2cm, bottom=2cm, left=2in, right=2in, textwidth=17cm, 
			 textheight=26cm, marginparwidth=0in, nofoot, centering}
\maketitle
\restoregeometry

\dominitoc
\dominilof
\dominilot
\tableofcontents

\pagenumbering{arabic}
\setcounter{page}{1}

% \LARGE

\chapter{Introduction}\label{ch:introduction}
\section{Balham Ukulele Society} % (fold)
\label{sec:balham_ukulele_society}

\paragraph{Welcome} % (fold)
\label{par:welcome}
We meet for a jam session every other Sunday at 7pm, in Balham Bowls Club. This is our official songbook, updated for 2012 and available to you, free and open source.
% paragraph welcome (end)

\paragraph{Adding to the song book} % (fold)
\label{par:adding_to_the_song_book}

New songs are always welcome, and easily added, if you know \LaTeX. Fork the repository from Github - http://github.com/vishvish/songbook - and amend away. Corrections are gratefully accepted via a pull request, and great new songs too.
% paragraph adding_to_the_song_book (end)

% section balham_ukulele_society (end)

\chapter{Songs / A -- M}
\label{ch:songs_a_m}
\minitoc
\Large
\linenumbers

\section{Ace Of Spades / Motorhead}\label{sec:ace_of_spades}
{\small (Really helps to play this with barre chords but your fingers will still get tired.)}
\Bmajor
\Cmajor
\Dmajor
\Emajor
\Gmajor

\upchord{E}Intro\\

\upchord{G}If you like to gamble, I tell you I'm your man

\upchord{G}You win some, lose some, it's all the same to me\upchord{E}

\upchord{D}The pleasure is to \upchord{C}play, it makes no difference what you say\upchord{E}

\upchord{D}I don't share your \upchord{C}greed, the only card I need is\\

(x2)\upchord{E}The Ace Of Spades

Alright\\

\upchord{G}Playing for the high one, dancing with the devil,

\upchord{G}Going with the flow, it's all a game to me\upchord{E}

Sev\upchord{D}en or \upchord{C}Eleven, snake eyes watching you\upchord{E}

\upchord{D}Double up or \upchord{C}quit, double stakes or splits\\

(x2)\upchord{E}The Ace Of Spades\\

\upchord{E}You know I'm born to lose, and \upchord{D}gamb\upchord{E}ling's for fools,

\upchord{E}But that's the way I like it babe

\upchord{E}I don't wanna live forever \upchord{D}\hrulefill\upchord{E}\hrulefill\upchord{D}\hrulefill\upchord{C}\hrulefill\upchord{B}\hrulefill

\upchord{B}And Don't Forget The Joker\upchord{E}\\

\upchord{G}Pushing up the ante, I know you've got to see me

\upchord{G}Read 'em and weep, the dead man's hand again\upchord{E}

\upchord{D}I see it in your \upchord{C}eyes, take one look and die\upchord{E}

\upchord{D}The only thing you \upchord{C}see, you know it's gonna be\\

(x2)\upchord{E}The Ace Of Spades



\section{All My Loving / The Beatles}\label{sec:all_my_loving}
{\small (Play Am/C by fingering both chords simultaneously. 3rd fret 1st string + 2nd fret 4th string)}

\Cmajor
\Gmajor
\Dminor
\Gseven
\Aminor
\Fmajor
\Bflat
\Caugmented

\upchord{C}Intro:\upchord{G}\hrulefill\upchord{C}\hrulefill

Close your \upchord{Dm}eyes and I'll \upchord{G7}kiss you

To\upchord{C}morrow I'll \upchord{Am}miss you

Re\upchord{F}member I'll \upchord{Dm}always be \upchord{B$\flat$}true 

\upchord{G7}And then \upchord{Dm}while I'm a\upchord{G7}way

I'll write \upchord{C}home every \upchord{Am}day

And I'll \upchord{F}send all my \upchord{G7}loving to \upchord{C}you\\

I'll pre\upchord{Dm}tend that I'm \upchord{G7}kissing

The \upchord{C}lips I am \upchord{Am}missing

And \upchord{F}hope that my \upchord{Dm}dreams will come \upchord{B$\flat$}true

\upchord{G7}And then \upchord{Dm}while I'm \upchord{G7}away

I'll write \upchord{C}home ev'ry \upchord{Am}day

And I'll \upchord{F}send all my \upchord{G7}loving to \upchord{C}you\\

\upchord{C}All my \upchord{Am/C}loving | \upchord{C+}I will send to \upchord{C}you 

\upchord{C}All my \upchord{Am/C}loving | \upchord{C+}darling I'll be \upchord{C}true\\

Instrumental: \upchord{F}\hrulefill\upchord{C}\hrulefill\upchord{Dm}\hrulefill\upchord{G7}\hrulefill\upchord{C}\hrulefill\\

Close your \upchord{Dm}eyes and I'll \upchord{G7}kiss you

To\upchord{C}morrow I'll \upchord{Am}miss you

Re\upchord{F}member I'll \upchord{Dm}always be \upchord{B$\flat$}true 

\upchord{G7}And then \upchord{Dm}while I'm a\upchord{G7}way

I'll write \upchord{C}home ev'ry \upchord{Am}day

And I'll \upchord{F}send all my \upchord{G7}loving to \upchord{C}you\\


All my \upchord{Am/C}loving | \upchord{C+}I will send to \upchord{C}you 

All my \upchord{Am/C}loving | \upchord{C+}darling I'll be \upchord{C}true 

All my \upchord{Am} loving all my \upchord{C}loving ooh

All my \upchord{Am}loving | I will send to \upchord{C}you

\section{Dirty Old Town / Ewan MacColl}\label{sec:dirty_old_town}
Info\footnote{You shouldn’t really need telling how to play this one. Note the unusual key differences: D, G and C). (For anyone who didn’t know Ewan MacColl was Kirsty MacColl's dad--Ewan MacColl was Kirsty MacColl's dad)}
\Cmajor
\Dmajor
\Fmajor
\Gmajor
\Aminor
\Bminor
\EminorSeven

\upchord{D}Intro:\upchord{G}\hrulefill\upchord{D}\hrulefill\upchord{Em7}\hrulefill\upchord{Bm}\hrulefill

I met my \upchord{G}love by the gas works wall 

Dreamed a \upchord{C}dream by the old ca\upchord{G}nal 

I kissed my girl by the factory wall

Dirty old \upchord{D}town

Dirty old \upchord{Em7}town\\


Clouds are \upchord{G}drifting across the moon 

Cats are \upchord{C}prowling on their \upchord{G}beat 

Spring's a girl from the streets at night 

Dirty old \upchord{D}town | Dirty old \upchord{Em7}town\\


Instrumental: \upchord{C}\hrulefill\upchord{F}\hrulefill\upchord{C}\hrulefill\upchord{G}\hrulefill\upchord{Am}\hrulefill\\


I heard a \upchord{G}siren from the docks

Saw a \upchord{C}train set the night on \upchord{G}fire 

I smelled the spring on the smoky wind 

Dirty old \upchord{D}town

Dirty old \upchord{Em7}town\\


I'm gonna \upchord{G}make me a big sharp axe 

Shining \upchord{C}steel tempered in the \upchord{G}fire 

I'll chop you down like an old dead tree 

Dirty old \upchord{D}town

Dirty old \upchord{Em7}town\\


I met my \upchord{G}love by the gas works wall 

Dreamed a \upchord{C}dream by the old ca\upchord{G}nal

I kissed my girl by the factory wall

Dirty old \upchord{Am}town

Dirty old \upchord{Em7}town

Dirty old \upchord{D}town

Dirty old \upchord{Em7}town
\section{Echo Beach / Martha and the Muffins}\label{sec:echo_beach}
\Cmajor
\Dmajor
\Fmajor
\Gmajor
\Aminor
\Eminor
\Bflat

\upchord{Am}Intro:\hrulefill\upchord{G}\hrulefill\upchord{Em}\hrulefill\upchord{F}\hrulefill\upchord{G}\hrulefill x 4

I \upchord{Am}know it's out of fashion, \upchord{D}and a \upchord{C}trifle \upchord{Am}uncool\upchord{D}\hrulefill\upchord{Em}\hrulefill

But \upchord{Am}I can't help it, \upchord{D}I'm a \upchord{C}romantic \upchord{Am}fool\upchord{D}\hrulefill\upchord{Em}\hrulefill

It's a \upchord{Am}habit of mine \upchord{D}to watch the \upchord{C}sun go \upchord{Am}down\upchord{D}\hrulefill

\upchord{Em}On \upchord{Am}Echo Beach, \upchord{D}I watch the \upchord{C}sun go \upchord{Am}down\upchord{D}\hrulefill\upchord{Em}\hrulefill\\

From \upchord{G}9 to 5 I have to spend my \upchord{D}time at work.

My \upchord{G}job is very boring, I'm an \upchord{D}office clerk.

The \upchord{Am}only thing that helps me pass the \upchord{Em}time away

Is \upchord{Am}knowing I'll be back in Echo \upchord{Em}Beach some day.\\

Instrumental:\upchord{Am}\hrulefill\upchord{G}\hrulefill\upchord{Em}\hrulefill\upchord{F}\hrulefill\upchord{G}\hrulefill x 4\\

On \upchord{Am}silent summer evenings \upchord{D}the sky's \upchord{C}alive with \upchord{Am}light\upchord{D}\hrulefill\upchord{Em}\hrulefill

A \upchord{Am}building in the distance - \upchord{D}surreal\upchord{C}istic \upchord{Am}sight\upchord{D}\hrulefill

On \upchord{Am}Echo Beach, \upchord{D}waves make the \upchord{C}only \upchord{Am}sound\upchord{D}\hrulefill

On \upchord{Am}Echo Beach, \upchord{D}there's not a \upchord{C}soul a\upchord{Am}round\upchord{D}\hrulefill\upchord{Em}\hrulefill\\

From \upchord{G}9 to 5 I have to spend my \upchord{D}time at work.

My \upchord{G}job is very boring, I'm an \upchord{D}office clerk.

The \upchord{Am}only thing that helps me pass the \upchord{Em}time away

Is \upchord{Am}knowing I'll be back in Echo \upchord{Em}Beach some day.\\

\upchord{F}\hrulefill\upchord{G}\hrulefill\upchord{B$\flat$}\hrulefill\upchord{C}\hrulefill x 2\\

\upchord{Am}Echo Beach \upchord{G}far away in time, \upchord{Em}Echo Beach \upchord{F}far away \upchord{G}in time...
(repeat to end)
\section{Five Foot Two / Ray Henderson}\label{ch:five_foot_two}
\Cmajor
\Aseven
\Dseven
\Eseven
\Gseven
\Gaugmented

\upchord{C}Five foot two, \upchord{E7}eyes of blue

\upchord{A7}But oh what those five foot could do

Has \upchord{D7}anybody \upchord{G7}seen my \upchord{C}girl?\\


\upchord{C}Turned up nose, \upchord{E7}turned down hose

\upchord{A7}Never had no other beaus

Has \upchord{D7}anybody \upchord{G7}seen my \upchord{C}girl?\\


\upchord{E7}Now if you run into a five foot two

\upchord{A7}All covered in fur

\upchord{D7}Diamond rings and all those things

\upchord{G7}Betcha life \upchord{D7}it isn't \upchord{G7}her\\


\upchord{G+}But

\upchord{C}Could she love, \upchord{E7}could she woo?

\upchord{A7}Could she, could she, could she coo?

Has \upchord{D7}anybody \upchord{G7}seen my \upchord{C}girl?

\section{Hallelujah / Leonard Cohen}\label{sec:hallelujah}
\Cmajor
\Fmajor
\Gmajor
\Aminor
\Eseven

\upchord{C}Intro:\upchord{Am}\hrulefill\upchord{C}\hrulefill\upchord{Am}

I \upchord{C}heard there was a \upchord{Am}secret chord

that \upchord{C}David played and it \upchord{Am}pleased the lord

but \upchord{F}you don't really \upchord{G}care for music \upchord{C}do you \upchord{G}

Well it \upchord{C}goes like this the \upchord{F}fourth, the \upchord{G}fifth a the \upchord{Am}minor fall and the \upchord{F}major lift | the \upchord{G}baffled king \upchord{E7}composing \upchord{Am}hallelujah

Halle-\upchord{F}lujah, Halle-\upchord{Am}lujah, Halle-\upchord{F}lujah, Hallelu-\upchord{C}\upchord{G}jah \upchord{C}\hrulefill\upchord{Am}\hrulefill\upchord{C}\hrulefill\upchord{Am}\hrulefill

Well your \upchord{C}faith was strong but you \upchord{Am}needed proof

You \upchord{C}saw her bathing \upchord{Am}on the roof

Her \upchord{F}beauty and the \upchord{G}moonlight overthrew \upchord{C}you \upchord{G}

She \upchord{C}tied you to her \upchord{F}kitchen \upchord{G}chair, she \upchord{Am}broke your throne and she \upchord{F}cut your hair | And \upchord{G}from your lips she \upchord{E7}drew the \upchord{Am}hallelujah

\upchord{C}Baby i've been \upchord{Am}here before I've \upchord{C}seen this room and I've \upchord{Am}walked this floor

I \upchord{F}used to live \upchord{G}alone before I \upchord{C}knew you \upchord{G}

I've \upchord{C}seen your flag on the \upchord{F}marble \upchord{G}arch

But \upchord{Am}love is not a \upchord{F}victory march

It's a \upchord{G}cold and it's a \upchord{E7}broken \upchord{Am}hallelujah

Well \upchord{C}there was a time when you \upchord{Am}let me know 

What's \upchord{C}really going \upchord{Am}on below

But \upchord{F}now you never \upchord{G}show that to me \upchord{C}do you \upchord{G}\hrulefill

But \upchord{C}remember when I \upchord{F}moved in \upchord{G}you | And the \upchord{Am}holy dove was \upchord{F}moving too

And \upchord{G}every breath we \upchord{E7}drew was \upchord{Am}hallelujah

Well \upchord{C}maybe there's a \upchord{Am}god above | But \upchord{C}all I've ever \upchord{Am}learned from love

Was \upchord{F}how to shoot \upchord{G}somebody who \upchord{C}outdrew you | \upchord{G}It's \upchord{C}not a cry that you \upchord{F}hear at \upchord{G}night

It's \upchord{Am}not somebody who's \upchord{F}seen the light | It's a \upchord{G}cold and it's a \upchord{E7}broken \upchord{Am}hallelujah

\upchord{F}Hallelujah, \upchord{Am}Hallelujah, \upchord{F}Hallelujah, \upchord{C}Hallel\upchord{G}ujah \upchord{C}\hrulefill

\nolinenumbers
% chapter songs_a_z (end)

\chapter{Songs / M -- Z} % (fold)
\label{cha:songs_m_z}
\minitoc
\Large
\linenumbers

\section{The Shoop Shoop Song / Rudy Clark}\label{sec:shoop_shoop_song}
% {\small(Girls: don't sing the parts in italics -- Guys: italics are for you)}
\Cmajor
\Fmajor
\Gmajor
\Aminor
\Dminor
\Dseven
\Eseven


\upchord{G}Does he love me \upchord{F}I want to know

\upchord{G}How can I tell if he loves me so

\emph{(Is it \upchord{Dm}in his \upchord{G}eyes?)}

Oh \upchord{Dm}no, you'll be de\upchord{G}ceived

\emph{(Is it \upchord{Dm}in his \upchord{G}eyes?)}

Oh \upchord{Dm}no he'll make be\upchord{G}lieve

If you \upchord{C}wanna \upchord{Am}know if \upchord{F}he loves you \upchord{G}so, it's in his \upchord{C}kiss\\


\emph{\upchord{F}(That's where it \upchord{G}is)} ...Oh yeah\\


\emph{(Is it \upchord{Dm}in his \upchord{G}face?)}

Oh \upchord{Dm}no, that's just his \upchord{G}charm

\emph{(In his \upchord{Dm}warm em\upchord{G}brace?)}

Oh \upchord{Dm}no, that's just his \upchord{G}arms

If you \upchord{C}wanna \upchord{Am}know if \upchord{F}he loves you \upchord{G}so, it's in his \upchord{C}kiss 

\upchord{F}(That's where it \upchord{G}is) ...Oh yeah

It's in his \upchord{C}kiss\\


\emph{\upchord{F}(That's where it \upchord{C}is)} Oh, Oh, Oh,\\


\upchord{E7}hug him, squeeze him tight

To \upchord{Am}find out what you want to know \upchord{D7}If it's love, if it really is

\upchord{G}It's there in his kiss

\emph{(How 'bout the \upchord{Dm}way he \upchord{G}acts)}

Oh \upchord{Dm}no, that's not the \upchord{G}way

\upchord{Dm}You're not \upchord{G}listening to \upchord{Dm}all I \upchord{G}say

If you \upchord{C}wanna \upchord{Am}know if \upchord{F}he loves you \upchord{G}so, it's in his \upchord{C}kiss\\


\emph{\upchord{F}(That's where it \upchord{G}is)}

Oh, oh, it's in his \upchord{C}kiss\\


\emph{\upchord{F}(That's where it \upchord{G}is)}\\


Instrumental:\upchord{Dm}\hrulefill\upchord{G}\hrulefill\upchord{Dm}\hrulefill\upchord{G}\hrulefill\upchord{C}\hrulefill x2\\


\upchord{E7}hug him, squeeze him tight

To \upchord{Am}find out what you want to know \upchord{D7}If it's love, if it really is

\upchord{G}It's there in his kiss

\emph{(How 'bout the \upchord{Dm}way he \upchord{G}acts)}

Oh \upchord{Dm}no, that's not the \upchord{G}way

\upchord{Dm}You're not \upchord{G}listening to \upchord{Dm}all I \upchord{G}say

If you \upchord{C}wanna \upchord{Am}know if \upchord{F}he loves you \upchord{G}so, it's in his \upchord{C}kiss\\
 

\emph{\upchord{F}(That's where it \upchord{G}is)}

Oh, oh, it's in his \upchord{C}kiss\\


\emph{\upchord{F}(That's where it \upchord{G}is)}

Oh, oh, it's in his \upchord{C}kiss\\


\emph{\upchord{F}(That's where it \upchord{G}is)}

Oh yeah, it's in his kiss\upchord{C}\hrulefill\upchord{F}\hrulefill\upchord{C}
\section{The Times They Are a-Changin' / Bob Dylan}\label{sec:times_they_are_a_changin}
\Cmajor
\Dmajor
\Gmajor
\Aminor
\Eminor

Come \upchord{G}gather round \upchord{Em}people \upchord{C}wherever you \upchord{G}roam

And \upchord{G}admit that the \upchord{Em}waters \upchord{C}around you have \upchord{D}grown

And \upchord{G}accept it that \upchord{Em}soon you'll be \upchord{C}drenched to the \upchord{G}bone

If your \upchord{G}time to \upchord{Am}you is worth \upchord{D}savin'

So you \upchord{D}better start \upchord{C}swimming or you'll \upchord{G}sink like a \upchord{D}stone

For the \upchord{G}times, they \upchord{C}are a-\upchord{D}chang-\upchord{G}in'\\


Come \upchord{G}writers and \upchord{Em}critics who \upchord{C}prophesise with your \upchord{G}pen 

And \upchord{G}keep your eyes \upchord{Em}wide the chance \upchord{C}won't come \upchord{D}again

And \upchord{G}don't speak too \upchord{Em}soon for the wheel's \upchord{C}still in \upchord{G}spin

And there's \upchord{G}no tellin' \upchord{Am}who that it's \upchord{D}namin'

For the \upchord{D}loser \upchord{C}now will be \upchord{G}later to \upchord{D}win

For the \upchord{G}times they \upchord{C}are a-\upchord{D}chang\upchord{G}in'\\


Come \upchord{G}mothers and \upchord{Em}fathers \upchord{C}throughout the \upchord{G}land

And \upchord{G}don't criti\upchord{Em}cize what you \upchord{C}don't under \upchord{D}stand

Your \upchord{G}sons and your \upchord{Em}daughters are \upchord{C}beyond your \upchord{G}command

Your \upchord{G}old road is \upchord{Am}rapidly \upchord{D}agin'

Please \upchord{D}get out of the \upchord{C}new one if you \upchord{G}can't lend a \upchord{D}hand

For the \upchord{G}times they \upchord{C}are a- \upchord{D}chan\upchord{G}gin'\\


Come \upchord{G}senators, \upchord{Em}congressmen \upchord{C}please heed the \upchord{G}call

Don't \upchord{G}stand in the \upchord{Em}doorway, don't \upchord{C}block up the \upchord{D}hall

For \upchord{G}he that gets \upchord{Em}hurt will be \upchord{C}he who has \upchord{G}stalled

There's a \upchord{G}battle out \upchord{Am}side and it's \upchord{D}ragin'

It'll \upchord{D}soon shake your \upchord{C}windows and \upchord{G}rattle your \upchord{D}walls

For the \upchord{G}times they \upchord{C}are a- \upchord{D}chang\upchord{G}in'\\


The \upchord{G}line it is \upchord{Em}drawn the \upchord{C}curse it is \upchord{G}cast

The \upchord{G}slow one \upchord{Em}now will \upchord{C}later be \upchord{D}fast

As the \upchord{G}present \upchord{Em}now will \upchord{C}later be \upchord{G}past

The \upchord{G}order is \upchord{Am}rapidly \upchord{D}fadin'

And the \upchord{D}first one \upchord{C}now will \upchord{G}later be \upchord{D}last

For the \upchord{G}times they \upchord{C}are a- \upchord{D}chang\upchord{G}in'

\nolinenumbers
% chapter songs_m_z (end)

\chapter{Appendix A: New Songs} % (fold)
\label{prt:appendix_a_new_songs}
\minitoc
\Large
\linenumbers
% \section{Ace Of Spades / Motorhead}\label{sec:ace_of_spades}
{\small (Really helps to play this with barre chords but your fingers will still get tired.)}
\Bmajor
\Cmajor
\Dmajor
\Emajor
\Gmajor

\upchord{E}Intro\\

\upchord{G}If you like to gamble, I tell you I'm your man

\upchord{G}You win some, lose some, it's all the same to me\upchord{E}

\upchord{D}The pleasure is to \upchord{C}play, it makes no difference what you say\upchord{E}

\upchord{D}I don't share your \upchord{C}greed, the only card I need is\\

(x2)\upchord{E}The Ace Of Spades

Alright\\

\upchord{G}Playing for the high one, dancing with the devil,

\upchord{G}Going with the flow, it's all a game to me\upchord{E}

Sev\upchord{D}en or \upchord{C}Eleven, snake eyes watching you\upchord{E}

\upchord{D}Double up or \upchord{C}quit, double stakes or splits\\

(x2)\upchord{E}The Ace Of Spades\\

\upchord{E}You know I'm born to lose, and \upchord{D}gamb\upchord{E}ling's for fools,

\upchord{E}But that's the way I like it babe

\upchord{E}I don't wanna live forever \upchord{D}\hrulefill\upchord{E}\hrulefill\upchord{D}\hrulefill\upchord{C}\hrulefill\upchord{B}\hrulefill

\upchord{B}And Don't Forget The Joker\upchord{E}\\

\upchord{G}Pushing up the ante, I know you've got to see me

\upchord{G}Read 'em and weep, the dead man's hand again\upchord{E}

\upchord{D}I see it in your \upchord{C}eyes, take one look and die\upchord{E}

\upchord{D}The only thing you \upchord{C}see, you know it's gonna be\\

(x2)\upchord{E}The Ace Of Spades



% \section{All My Loving / The Beatles}\label{sec:all_my_loving}
{\small (Play Am/C by fingering both chords simultaneously. 3rd fret 1st string + 2nd fret 4th string)}

\Cmajor
\Gmajor
\Dminor
\Gseven
\Aminor
\Fmajor
\Bflat
\Caugmented

\upchord{C}Intro:\upchord{G}\hrulefill\upchord{C}\hrulefill

Close your \upchord{Dm}eyes and I'll \upchord{G7}kiss you

To\upchord{C}morrow I'll \upchord{Am}miss you

Re\upchord{F}member I'll \upchord{Dm}always be \upchord{B$\flat$}true 

\upchord{G7}And then \upchord{Dm}while I'm a\upchord{G7}way

I'll write \upchord{C}home every \upchord{Am}day

And I'll \upchord{F}send all my \upchord{G7}loving to \upchord{C}you\\

I'll pre\upchord{Dm}tend that I'm \upchord{G7}kissing

The \upchord{C}lips I am \upchord{Am}missing

And \upchord{F}hope that my \upchord{Dm}dreams will come \upchord{B$\flat$}true

\upchord{G7}And then \upchord{Dm}while I'm \upchord{G7}away

I'll write \upchord{C}home ev'ry \upchord{Am}day

And I'll \upchord{F}send all my \upchord{G7}loving to \upchord{C}you\\

\upchord{C}All my \upchord{Am/C}loving | \upchord{C+}I will send to \upchord{C}you 

\upchord{C}All my \upchord{Am/C}loving | \upchord{C+}darling I'll be \upchord{C}true\\

Instrumental: \upchord{F}\hrulefill\upchord{C}\hrulefill\upchord{Dm}\hrulefill\upchord{G7}\hrulefill\upchord{C}\hrulefill\\

Close your \upchord{Dm}eyes and I'll \upchord{G7}kiss you

To\upchord{C}morrow I'll \upchord{Am}miss you

Re\upchord{F}member I'll \upchord{Dm}always be \upchord{B$\flat$}true 

\upchord{G7}And then \upchord{Dm}while I'm a\upchord{G7}way

I'll write \upchord{C}home ev'ry \upchord{Am}day

And I'll \upchord{F}send all my \upchord{G7}loving to \upchord{C}you\\


All my \upchord{Am/C}loving | \upchord{C+}I will send to \upchord{C}you 

All my \upchord{Am/C}loving | \upchord{C+}darling I'll be \upchord{C}true 

All my \upchord{Am} loving all my \upchord{C}loving ooh

All my \upchord{Am}loving | I will send to \upchord{C}you

% \section{Dirty Old Town / Ewan MacColl}\label{sec:dirty_old_town}
Info\footnote{You shouldn’t really need telling how to play this one. Note the unusual key differences: D, G and C). (For anyone who didn’t know Ewan MacColl was Kirsty MacColl's dad--Ewan MacColl was Kirsty MacColl's dad)}
\Cmajor
\Dmajor
\Fmajor
\Gmajor
\Aminor
\Bminor
\EminorSeven

\upchord{D}Intro:\upchord{G}\hrulefill\upchord{D}\hrulefill\upchord{Em7}\hrulefill\upchord{Bm}\hrulefill

I met my \upchord{G}love by the gas works wall 

Dreamed a \upchord{C}dream by the old ca\upchord{G}nal 

I kissed my girl by the factory wall

Dirty old \upchord{D}town

Dirty old \upchord{Em7}town\\


Clouds are \upchord{G}drifting across the moon 

Cats are \upchord{C}prowling on their \upchord{G}beat 

Spring's a girl from the streets at night 

Dirty old \upchord{D}town | Dirty old \upchord{Em7}town\\


Instrumental: \upchord{C}\hrulefill\upchord{F}\hrulefill\upchord{C}\hrulefill\upchord{G}\hrulefill\upchord{Am}\hrulefill\\


I heard a \upchord{G}siren from the docks

Saw a \upchord{C}train set the night on \upchord{G}fire 

I smelled the spring on the smoky wind 

Dirty old \upchord{D}town

Dirty old \upchord{Em7}town\\


I'm gonna \upchord{G}make me a big sharp axe 

Shining \upchord{C}steel tempered in the \upchord{G}fire 

I'll chop you down like an old dead tree 

Dirty old \upchord{D}town

Dirty old \upchord{Em7}town\\


I met my \upchord{G}love by the gas works wall 

Dreamed a \upchord{C}dream by the old ca\upchord{G}nal

I kissed my girl by the factory wall

Dirty old \upchord{Am}town

Dirty old \upchord{Em7}town

Dirty old \upchord{D}town

Dirty old \upchord{Em7}town
% \section{Echo Beach / Martha and the Muffins}\label{sec:echo_beach}
\Cmajor
\Dmajor
\Fmajor
\Gmajor
\Aminor
\Eminor
\Bflat

\upchord{Am}Intro:\hrulefill\upchord{G}\hrulefill\upchord{Em}\hrulefill\upchord{F}\hrulefill\upchord{G}\hrulefill x 4

I \upchord{Am}know it's out of fashion, \upchord{D}and a \upchord{C}trifle \upchord{Am}uncool\upchord{D}\hrulefill\upchord{Em}\hrulefill

But \upchord{Am}I can't help it, \upchord{D}I'm a \upchord{C}romantic \upchord{Am}fool\upchord{D}\hrulefill\upchord{Em}\hrulefill

It's a \upchord{Am}habit of mine \upchord{D}to watch the \upchord{C}sun go \upchord{Am}down\upchord{D}\hrulefill

\upchord{Em}On \upchord{Am}Echo Beach, \upchord{D}I watch the \upchord{C}sun go \upchord{Am}down\upchord{D}\hrulefill\upchord{Em}\hrulefill\\

From \upchord{G}9 to 5 I have to spend my \upchord{D}time at work.

My \upchord{G}job is very boring, I'm an \upchord{D}office clerk.

The \upchord{Am}only thing that helps me pass the \upchord{Em}time away

Is \upchord{Am}knowing I'll be back in Echo \upchord{Em}Beach some day.\\

Instrumental:\upchord{Am}\hrulefill\upchord{G}\hrulefill\upchord{Em}\hrulefill\upchord{F}\hrulefill\upchord{G}\hrulefill x 4\\

On \upchord{Am}silent summer evenings \upchord{D}the sky's \upchord{C}alive with \upchord{Am}light\upchord{D}\hrulefill\upchord{Em}\hrulefill

A \upchord{Am}building in the distance - \upchord{D}surreal\upchord{C}istic \upchord{Am}sight\upchord{D}\hrulefill

On \upchord{Am}Echo Beach, \upchord{D}waves make the \upchord{C}only \upchord{Am}sound\upchord{D}\hrulefill

On \upchord{Am}Echo Beach, \upchord{D}there's not a \upchord{C}soul a\upchord{Am}round\upchord{D}\hrulefill\upchord{Em}\hrulefill\\

From \upchord{G}9 to 5 I have to spend my \upchord{D}time at work.

My \upchord{G}job is very boring, I'm an \upchord{D}office clerk.

The \upchord{Am}only thing that helps me pass the \upchord{Em}time away

Is \upchord{Am}knowing I'll be back in Echo \upchord{Em}Beach some day.\\

\upchord{F}\hrulefill\upchord{G}\hrulefill\upchord{B$\flat$}\hrulefill\upchord{C}\hrulefill x 2\\

\upchord{Am}Echo Beach \upchord{G}far away in time, \upchord{Em}Echo Beach \upchord{F}far away \upchord{G}in time...
(repeat to end)
% \section{Five Foot Two / Ray Henderson}\label{ch:five_foot_two}
\Cmajor
\Aseven
\Dseven
\Eseven
\Gseven
\Gaugmented

\upchord{C}Five foot two, \upchord{E7}eyes of blue

\upchord{A7}But oh what those five foot could do

Has \upchord{D7}anybody \upchord{G7}seen my \upchord{C}girl?\\


\upchord{C}Turned up nose, \upchord{E7}turned down hose

\upchord{A7}Never had no other beaus

Has \upchord{D7}anybody \upchord{G7}seen my \upchord{C}girl?\\


\upchord{E7}Now if you run into a five foot two

\upchord{A7}All covered in fur

\upchord{D7}Diamond rings and all those things

\upchord{G7}Betcha life \upchord{D7}it isn't \upchord{G7}her\\


\upchord{G+}But

\upchord{C}Could she love, \upchord{E7}could she woo?

\upchord{A7}Could she, could she, could she coo?

Has \upchord{D7}anybody \upchord{G7}seen my \upchord{C}girl?

% \section{Hallelujah / Leonard Cohen}\label{sec:hallelujah}
\Cmajor
\Fmajor
\Gmajor
\Aminor
\Eseven

\upchord{C}Intro:\upchord{Am}\hrulefill\upchord{C}\hrulefill\upchord{Am}

I \upchord{C}heard there was a \upchord{Am}secret chord

that \upchord{C}David played and it \upchord{Am}pleased the lord

but \upchord{F}you don't really \upchord{G}care for music \upchord{C}do you \upchord{G}

Well it \upchord{C}goes like this the \upchord{F}fourth, the \upchord{G}fifth a the \upchord{Am}minor fall and the \upchord{F}major lift | the \upchord{G}baffled king \upchord{E7}composing \upchord{Am}hallelujah

Halle-\upchord{F}lujah, Halle-\upchord{Am}lujah, Halle-\upchord{F}lujah, Hallelu-\upchord{C}\upchord{G}jah \upchord{C}\hrulefill\upchord{Am}\hrulefill\upchord{C}\hrulefill\upchord{Am}\hrulefill

Well your \upchord{C}faith was strong but you \upchord{Am}needed proof

You \upchord{C}saw her bathing \upchord{Am}on the roof

Her \upchord{F}beauty and the \upchord{G}moonlight overthrew \upchord{C}you \upchord{G}

She \upchord{C}tied you to her \upchord{F}kitchen \upchord{G}chair, she \upchord{Am}broke your throne and she \upchord{F}cut your hair | And \upchord{G}from your lips she \upchord{E7}drew the \upchord{Am}hallelujah

\upchord{C}Baby i've been \upchord{Am}here before I've \upchord{C}seen this room and I've \upchord{Am}walked this floor

I \upchord{F}used to live \upchord{G}alone before I \upchord{C}knew you \upchord{G}

I've \upchord{C}seen your flag on the \upchord{F}marble \upchord{G}arch

But \upchord{Am}love is not a \upchord{F}victory march

It's a \upchord{G}cold and it's a \upchord{E7}broken \upchord{Am}hallelujah

Well \upchord{C}there was a time when you \upchord{Am}let me know 

What's \upchord{C}really going \upchord{Am}on below

But \upchord{F}now you never \upchord{G}show that to me \upchord{C}do you \upchord{G}\hrulefill

But \upchord{C}remember when I \upchord{F}moved in \upchord{G}you | And the \upchord{Am}holy dove was \upchord{F}moving too

And \upchord{G}every breath we \upchord{E7}drew was \upchord{Am}hallelujah

Well \upchord{C}maybe there's a \upchord{Am}god above | But \upchord{C}all I've ever \upchord{Am}learned from love

Was \upchord{F}how to shoot \upchord{G}somebody who \upchord{C}outdrew you | \upchord{G}It's \upchord{C}not a cry that you \upchord{F}hear at \upchord{G}night

It's \upchord{Am}not somebody who's \upchord{F}seen the light | It's a \upchord{G}cold and it's a \upchord{E7}broken \upchord{Am}hallelujah

\upchord{F}Hallelujah, \upchord{Am}Hallelujah, \upchord{F}Hallelujah, \upchord{C}Hallel\upchord{G}ujah \upchord{C}\hrulefill
% \section{The Shoop Shoop Song / Rudy Clark}\label{sec:shoop_shoop_song}
% {\small(Girls: don't sing the parts in italics -- Guys: italics are for you)}
\Cmajor
\Fmajor
\Gmajor
\Aminor
\Dminor
\Dseven
\Eseven


\upchord{G}Does he love me \upchord{F}I want to know

\upchord{G}How can I tell if he loves me so

\emph{(Is it \upchord{Dm}in his \upchord{G}eyes?)}

Oh \upchord{Dm}no, you'll be de\upchord{G}ceived

\emph{(Is it \upchord{Dm}in his \upchord{G}eyes?)}

Oh \upchord{Dm}no he'll make be\upchord{G}lieve

If you \upchord{C}wanna \upchord{Am}know if \upchord{F}he loves you \upchord{G}so, it's in his \upchord{C}kiss\\


\emph{\upchord{F}(That's where it \upchord{G}is)} ...Oh yeah\\


\emph{(Is it \upchord{Dm}in his \upchord{G}face?)}

Oh \upchord{Dm}no, that's just his \upchord{G}charm

\emph{(In his \upchord{Dm}warm em\upchord{G}brace?)}

Oh \upchord{Dm}no, that's just his \upchord{G}arms

If you \upchord{C}wanna \upchord{Am}know if \upchord{F}he loves you \upchord{G}so, it's in his \upchord{C}kiss 

\upchord{F}(That's where it \upchord{G}is) ...Oh yeah

It's in his \upchord{C}kiss\\


\emph{\upchord{F}(That's where it \upchord{C}is)} Oh, Oh, Oh,\\


\upchord{E7}hug him, squeeze him tight

To \upchord{Am}find out what you want to know \upchord{D7}If it's love, if it really is

\upchord{G}It's there in his kiss

\emph{(How 'bout the \upchord{Dm}way he \upchord{G}acts)}

Oh \upchord{Dm}no, that's not the \upchord{G}way

\upchord{Dm}You're not \upchord{G}listening to \upchord{Dm}all I \upchord{G}say

If you \upchord{C}wanna \upchord{Am}know if \upchord{F}he loves you \upchord{G}so, it's in his \upchord{C}kiss\\


\emph{\upchord{F}(That's where it \upchord{G}is)}

Oh, oh, it's in his \upchord{C}kiss\\


\emph{\upchord{F}(That's where it \upchord{G}is)}\\


Instrumental:\upchord{Dm}\hrulefill\upchord{G}\hrulefill\upchord{Dm}\hrulefill\upchord{G}\hrulefill\upchord{C}\hrulefill x2\\


\upchord{E7}hug him, squeeze him tight

To \upchord{Am}find out what you want to know \upchord{D7}If it's love, if it really is

\upchord{G}It's there in his kiss

\emph{(How 'bout the \upchord{Dm}way he \upchord{G}acts)}

Oh \upchord{Dm}no, that's not the \upchord{G}way

\upchord{Dm}You're not \upchord{G}listening to \upchord{Dm}all I \upchord{G}say

If you \upchord{C}wanna \upchord{Am}know if \upchord{F}he loves you \upchord{G}so, it's in his \upchord{C}kiss\\
 

\emph{\upchord{F}(That's where it \upchord{G}is)}

Oh, oh, it's in his \upchord{C}kiss\\


\emph{\upchord{F}(That's where it \upchord{G}is)}

Oh, oh, it's in his \upchord{C}kiss\\


\emph{\upchord{F}(That's where it \upchord{G}is)}

Oh yeah, it's in his kiss\upchord{C}\hrulefill\upchord{F}\hrulefill\upchord{C}

\nolinenumbers
% chapter appendix_a_new_songs (end)

\end{document}
\end

